\chapter{Vorwort}

\begin{quote}
	\glqq{}Schon vor 60 Jahren haben wir am CERN unsere Daten mit Neuronalen Netzen ausgewertet. Damals noch mit Stift und Papier und sie hießen Physiker.\grqq{}
\end{quote}

Dieses Zitat wurde während einem Vortrag über die Arbeit der Physiker am CERN im Jahre 2022 erwähnt. Auch wenn dieser Satz eine humorvolle Anspielung auf die Arbeitsweisen von Physikerinnen und Physikern in den 1960er Jahren war, die neben Stift und Papier selten bis gar nicht die Möglichkeit hatten, die Rechenmaschinen der damaligen Zeit zu verwenden, hat sich an machen Stellen an der Arbeitsweise der Physikerinnen und Physiker nicht viel Geändert. In den Universitäten wird immer noch gelehrt, wie dieselben unveränderten Rechnungen von Hand gelöst werden. Dabei ist der Taschenrechner aus dem heutigen Haushalt, sei es noch ein eigenständiges Gerät oder eine smartphone-App, nicht mehr weg zu denken. Doch die Taschenrechner für die Physik müssen teilweise noch selber gebaut werden, da unsere mathematischen Probleme bei weitem den Funktionsumfang des handelsüblichen Taschenrechners für die vier Grundrechenarten sprengen. \\
Ein so gebauter Taschenrechner muss aber kein Hexenwerk sein. Durch die immer weiter steigende Anzahl an Open-Source Werkzeuggen für viele Programmiersprachen fehlt nur noch das Wissen, wie man diese richtig kombiniert. Natürlich ist es aber auch wichtig, nicht einfach nur blind ein paar Bibliotheken zu vertrauen, ohne zu wissen wie diese intern arbeiten sondern auch die Konzepte dahinter zu lernen. Ein Open-source Tool soll schließlich keine Black box sein, deren interner Aufbau für die meisten Anwenderinnen und Anwender unbekannt ist sondern wie ein Uhrwerk in einem Glaskasten seine extraordinäre Arbeitsweise preisgeben. Letzteres ermöglicht es auch, dass Baufehler die dazu führen, dass das Werkzeug nicht richtig funktioniert, von aller Welt erkannt, gemeldet und vielleicht auch von fleißigen Freiwilligen behoben werden. \\
\textbf{Dies ist kein Python-Buch!} Na ja, irgendwie ist es das doch, denn alle Bücher die über die Programmiersprache Python lehren sind Python-Bücher. Aber dieses Buch vermittelt keinen Einstieg in die Sprache Python, Grundkenntnisse werden vorausgesetzt. Stattdessen geht es hier darum, wie man die Physik mit modernen mitteln bestreitet: Python-Skripte schreiben, die im Vergleich zum Menschen keine Rechenfehler machen und auch noch deutlich schneller rechnen. Die Bausteine, die schon viel Arbeit abnehmen, sind meistens als Open-Source Bibliothek wie z\,B\, \textit{numpy} oder \textit{sympy} schon vorgegeben und selbst dir Grafische Darstellung von Messwerten lässt sich mit \textit{Matplotlib} einfach erledigen. Dennoch gehört es dazu, dass die darin verwendeten Algorithmen erklärt oder teilweise auch selber implementiert werden.
Das Ziel dieses Lektüre ist, dass die Leserinnen und Leser aus dem Jahrhunderte alten Muster heraus kommen und aufhören, Stift und Papier (oder Kreide und Tafel) dem Computer vorzuziehen.
