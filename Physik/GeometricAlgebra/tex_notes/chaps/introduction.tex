\chapter{Einführung}

\section{Eine Kombination aus Geometrie und Algebra}

\subsection{Euklidische Geometrie}

Die Lehre der Geometrie ist das Teilgebiet der Mathematik, das sich in zwei-, drei- oder beliebig-dimensionalen Räumen mit Punkten, Geraden, Abständen, Winkeln und anderen Objekten und deren Eigenschaften befasst. Objekte können in einem Raum eine Position oder eine Größe haben wobei als Größe eine Länge, ein Winkel, ein Fläche oder eine andere Eigenschaft bezeichnet wird, sich von der Position im Raum unterscheidet. Der erste Versuch, eine Grundlage für die Geometrie zu formulieren, stammt von dem Griechen \textit{Euklid von Alexandria}. In der \textit{Euklidischen Geometrie} werden über 23 Definitionen ein paar Grundbegriffe eingeführt. 5 Postulate und 5 Axiome beschreiben die Eigenschaften auf denen die Geometrie aufbaut.

\subsubsection{Die Definitionen}
\begin{enumerate}
	\item Ein \textbf{Punkt} hat keine Ausdehnung
	\item Eine \textbf{Linie} hat nur eine Länge aber keine Breite
	\item Die Enden einer Linie sind Punkte
	\item Eine Linie, die gleichmäßig zu den Punkten auf ihr liegt, ist \textbf{Gerade}
	\item Eine \textbf{Fläche} hat eine Länge und Breite
	\item Die Enden einer Fläche sind Linien
	\item Eine \textbf{Ebene} ist eine Fläche, die gleichmäßig zu den Linien auf ihr liegt
	\item Als \textbf{Ebener Winkel} wird die Neigung zweier Linien zueinander in einer Ebene bezeichnet
	\item Sind die Linien Geraden, heißt der Winkel \textbf{geradlinig}
	\item Sind alle Winkel an zwei sich schneidenden Geraden gleich, so sind die Winkel \textbf{Rechte}. Jede dieser Geraden heißt \textbf{senkrecht} zu der, auf der sie steht.
	\item Ein Winkel, der größer als ein Rechter ist, heißt \textbf{Stumpf}
	\item Ein Winkel, der kleiner als ein Rechter ist, heißt \textbf{Spitz}
	\item Eine \textbf{Grenze} beschreibt das Ende von einer Linie / Fläche
	\item Als \textbf{Figur} wir alles bezeichnet, was von Grenzen umfasst wird
	\item Ein \textbf{Kreis} ist eine ebene Figur die von einer Linie umfasst wird, sodass alle Strecken, die von einem Punkt im Inneren bis zur Linie laufen, gleich sind.
	\item  Der Punkt in der Mitte des Kreises heißt \textbf{Mittelpunkt}
	\item Zum \textbf{Durchmesser} eines Kreises gehört jede Strecke, die durch den Mittelpunkt geht und auf beiden Seiten vom Kreisumfang begrenzt ist. Eine solche Strecke halbiert den Kreis in zwei gleiche Teile.
	\item Der \textbf{Halbkreis} ist die vom Durchmesser und dem durch ihn abgeschnittenen Bogen umfasste Figur
	\item Figuren, die von Strecken umfasst sind, heißen \textbf{Geradlinig}: \textbf{Dreiseite} bestehen aus 3 Strecken, \textbf{Vierseite} bestehen aus 4 Strecken, \textbf{Vielseite} bestehen aus mehreren umfassenden Strecken
	\item Unter den Dreiseiten hat das \textbf{gleichseitige Dreieck} drei gleiche Seiten, das \textbf{gleichschenklige Dreieck} nur zwei gleiche Seiten und das \textbf{ungleichseitige Dreieck} drei verschiedene Seiten
	\item Unter den Dreiseiten gibt es auch das \textbf{rechtwinklige Dreieck}, das einen rechten Winkel hat, das \textbf{stumpfwinklige Dreieck} mit einem stumpfen Winkel und das \textbf{spitzwinklige Dreieck} das 3 spitze Winkel hat.
	\item Unter den Vierseiten ist das \textbf{Quadrat} sowohl gleichseitig als rechtwinklig. Ein \textbf{Rechteck} ist rechtwinklig, aber nicht Gegenwinklig, ohne rechtwinklig oder gleichseitig zu sein. Die andern Vierseite heißen \textbf{Trapeze}
	\item \textbf{Parallelen} sind Geraden, die in derselben Ebene liegen und sich auch bei Verlängerung nach beiden Seiten ins Unendliche nicht treffen.
\end{enumerate}

Die Definitionen dienen dazu, ein paar Begriffe einzuführen, damit verständlich ist, worauf die Postulate und Axiome sich beziehen. Aus heutiger Sicht sind diese Definitionen wenig hilfreich. Es wäre sinnvoller, zuerst die Begriffe die zur Beschreibung der Geometrie verwendet werden aufzulisten um dann später zu zeigen, welchen Regeln sie unterliegen anstatt die grafische Darstellung eines Objektes mit Worten zu beschreiben. Aus diesem Grund wurden die Definitionen hier umformuliert um verständlicher zu wirken. Die erste Definition hieß ursprünglich \textit{\glqq{}Ein Punkt ist das, was keine Teile hat\grqq{}}, woraus aber nicht wirklich ersichtlich wird, was ein Punkt eigentlich ist. Erst recht müsste dafür definiert sein, was ein Teil ist. Ähnliches gilt für die zweite Definition von der Linie: Sie ist von der Definition des Begriffs \textit{Breite} abhängig aber die Breite wurde von Euklid nicht definiert. Sinnvoller ist es, erst die Eigenschaften eines Punktes zu nennen und dann daraus abzuleiten, wie sich ein Punkt als grafisches Objekt darstellen lässt. Daher sind diese Definitionen für das Verständnis der Geometrie überflüssig, die Axiome und Postulate erfüllen alleine ihren Zweck als Grundlage der Geometrie.

\subsubsection{Die Axiome}
\begin{enumerate}
	\item Dinge, die zu dem selben gleich sind, sind auch untereinander gleich
	\item Wenn man zu gleichen Dingen Gleiches hinzufügt, ist das Ergebnis gleich
	\item Wenn man von gleichen Dingen Gleiches abzieht, sind die Reste gleich
	\item Dinge, die einander decken, sind gleich (Kongruenz)
	\item Das Ganze ist größer als ein Teil davon
\end{enumerate}

\subsubsection{Die Postulate}
\begin{enumerate}
	\item Man kann von jedem Punkt aus zu jedem anderen Punkt eine Linie ziehen
	\item Begrenzte Linien können zu einer Geraden verlängert werden
	\item Wenn man einen Mittelpunkt und einen Abstand vorgibt, kann man daraus einen Kreis zeichnen
	\item Alle rechten Winkel sind gleich groß
	\item Wenn bei einer Geraden, die zwei andere Geraden schneidet, die Summe der beiden Innenwinkel (Nachbarwinkel) an der gleichen Seite kleiner ist als die Summe von zwei rechten Winkeln, so werden sich die beiden Geraden auf der Seite schneiden, an der sich diese beiden Winkel befinden.
\end{enumerate}

Das letzte Postulat, auch als \textit{\glqq{}Parallelpostulat\grqq{}} bekannt, lässt sich auch einfacher formulieren: Wenn ein Punkt außerhalb einer Geraden vorgegeben ist, dann gibt es genau eine weitere Gerade die zu der vorgegebenen Gerade parallel ist und durch diesen Punkt verläuft. Das fünfte Postulat ist etwas länger und komplizierter formuliert als die anderen, weshalb versucht wurde, es aus den anderen 4 Postulaten herzuleiten und damit überflüssig zu machen. Jedoch ist es Gauß, Lobatschewski und Bolyai gelungen, zu zeigen, dass die euklidische Geometrie ohne dieses Postulat unvollständig ist.

\subsection{Nicht-Euklidische Geometrie}

Bei ihren Überlegungen zum \textit{Parallelpostulat} haben Gauß, Lobatschewski und Bolyai zwei Geometrien konstruiert, in denen das Parallelpostulat nicht gültig ist: Wenn der Raum auf einem Hyperboloid oder auf einer Kugeloberfläche liegt. Die Kugeloberfläche wird in Abbildung \ref{fig:nicht-euklidiche-Kugeloberfläche} dargestellt. Hier gilt für das fünfte Postulat:
\begin{quote}
Es gibt zu einer Geraden keine parallele Gerade, die durch einen Punkt außerhalb der Geraden verläuft.
\end{quote}

% erstelle ein PGF Bild einer Kugel mit Python Matplotlib um dieses darunter wieder in diesem LaTeX Dokument einzubinden
\begin{pycode}
import matplotlib as mpl
mpl.use('pgf')
import matplotlib.pyplot as plt
from matplotlib import ticker
import numpy as np
# But with fonts from the document body
plt.rcParams.update({
	"font.family": "serif",  # use serif/main font for text elements
	"text.usetex": True,     # use inline math for ticks
	"pgf.rcfonts": False     # don't setup fonts from rc parameters
	})
# plot a 3 dimensional sphere
plt.rcParams["figure.figsize"] = [7.00, 3.50]
plt.rcParams["figure.autolayout"] = True
fig = plt.figure()
ax = fig.add_subplot(projection='3d')
r = 0.05
u, v = np.mgrid[0:2 * np.pi:30j, 0:np.pi:20j]
x = np.cos(u) * np.sin(v)
y = np.sin(u) * np.sin(v)
z = np.cos(v)
ax.plot_surface(x, y, z, cmap=plt.cm.viridis)
#ax.set_axis_off()
ax.grid(True)
for axis in [ax.xaxis, ax.yaxis, ax.zaxis]:
	for tick in axis.get_major_ticks():
		tick.tick1line.set_visible(False)
		tick.tick2line.set_visible(False)
		tick.label1.set_visible(False)
		tick.label2.set_visible(False)
plt.savefig("figs/sphere.pgf", format="pgf")
\end{pycode}
\begin{figure}[h]
	\centering
	\makeatletter
	\@input{figs/sphere.pgf}
	\makeatother
	\caption{Die Oberfläche einer Kugel}
	\label{fig:nicht-euklidiche-Kugeloberfläche}
\end{figure}

Wenn auf einer Kugeloberfläche angefangen wird, zwei Linien zu zeichnen, dann werden diese Linien sich zwangsweise irgendwann schneiden. Dabei wird angenommen, dass sich eine Linie auf einer Kugeloberfläche ins Unendliche verlängern lässt obwohl sie sich an einem beliebigen Punkt selbst wieder berührt (z.B. Wird die Äquatoriallinie nach einer Umdrehung wieder auf sich selber liegen, was dennoch als eine unendliche Verlängerung akzeptiert wird).

Ganz anderen Regeln folgt eine hyperbolische Fläche, wie sie in Abbildung \ref{fig:nicht-euklidiche-Hyperbelfläche} zu sehen ist. Hier lautet das fünfte Postulat:
\begin{quote}
Es gibt zu einer Geraden mehrere parallele Gerade, die durch einen Punkt außerhalb der Geraden verlaufen.
\end{quote}

% erstelle ein PGF Bild einer hyperbolischen Fläche mit Python Matplotlib um dieses darunter wieder in diesem LaTeX Dokument einzubinden
\begin{pycode}
import matplotlib as mpl
mpl.use('pgf')
import matplotlib.pyplot as plt
from matplotlib import ticker
import numpy as np
# But with fonts from the document body
plt.rcParams.update({
	"font.family": "serif",  # use serif/main font for text elements
	"text.usetex": True,     # use inline math for ticks
	"pgf.rcfonts": False     # don't setup fonts from rc parameters
	})
# plot a 3 dimensional hyperbolic plane
plt.rcParams["figure.figsize"] = [7.00, 3.50]
plt.rcParams["figure.autolayout"] = True
fig = plt.figure()
ax = fig.add_subplot(projection='3d')
X = np.linspace(-5, 5, 100)
Y = np.linspace(-5, 5, 100)
X, Y = np.meshgrid(X, Y)
Z = np.sqrt(0.3*(X **2 + Y **2) + 1)
ax.plot_surface(X, Y, Z, cmap=plt.cm.viridis, vmin=-5, vmax=5)
#ax.set_axis_off()
ax.grid(True)
for axis in [ax.xaxis, ax.yaxis, ax.zaxis]:
	for tick in axis.get_major_ticks():
		tick.tick1line.set_visible(False)
		tick.tick2line.set_visible(False)
		tick.label1.set_visible(False)
		tick.label2.set_visible(False)
plt.savefig("figs/hyperbolia.pgf", format="pgf")
\end{pycode}
\begin{figure}[h]
	\centering
	\makeatletter
	\@input{figs/hyperbolia.pgf}
	\makeatother
	\caption{Eine hyperbolische Fläche}
	\label{fig:nicht-euklidiche-Hyperbelfläche}
\end{figure}

Durch die Krümmung der Oberfläche lassen sich mehrere Geraden zeichnen die durch einen Punkt außerhalb einer vorgegebenen Geraden verlaufen, sie aber nicht schneiden obwohl sie in der gekrümmten ebene liegen. Die Winkelsummen von Dreiecken in sphärischen und hyperbolischen Geometrien summieren sich nicht zu \SI{180}{\degree} da sie auf einer Kugeloberfläche größer und auf einer Hyperbel kleiner als \SI{180}{\degree} sein kann. Falls weiteres Interesse über euklidische und nicht euklidische Geometrie besteht, kann das folgende YouTube-Video über dieses Thema empfohlen werden: \url{https://www.youtube.com/watch?v=lFlu60qs7_4}.

\subsection{Eine Prise Algebra}

Bei der Algebra handelt es sich ebenfalls um ein früh entdecktes Teilgebiet der Mathematik. Mathematische Ausdrücke sind entweder wahr oder falsch. Daher können wir mit Sicherheit sagen, dass $3 + 4 = 7$ gilt. Diese Aussage gilt aber nur für die Zahlen, die wir hier verwendet haben und auch nur in dieser Reihenfolge. In der Algebra geht es darum, Ausdrücke zu formulieren, die für beliebige Zahlen gelten. Auf den Ausdruck $(3 + 4)^2 = 49$ lässt sich eine algebraische Regel, die binomische Formel, anwenden, weil der Ausdruck $(a + b)^2 = a^2 + 2 \cdot a \cdot b + b^2$ für beliebige Zahlen $a$ und $b$ gültig ist.

Lassen sich solche allgemein formulierbare Aussagen auch auf geometrische Objekte Übertragen? Gibt es eine Möglichkeit, Dreiecke miteinander oder mit Punkten zu addieren? Die Geometrische Algebra beschreibt eine Mathematik in der geometrische Objekte auf eine Art und Weise dargestellt werden, wie deren Kombinationen und Wechselwirkungen miteinander einfach bestimmt werden können, sodass sich damit vielleicht wirklich Abbildung \ref{fig:addition-dreiecke} ausdrücken lässt (wobei es sich bei dieser Skizze nur um eine symbolische Darstellung zur besseren Vorstellung handelt die keine Ansprüche auf Richtigkeit hat).

\begin{figure}
	\centering
	\begin{tikzpicture}
		\draw (0,0) coordinate (a1) --
					(2,0) coordinate (c1) --
					(2,2) coordinate (b1) --
					cycle;
		\node at (3,1) {+};
		\draw (4.5,0) coordinate (a2) --
					(7,1) coordinate (c2) --
					(4,2) coordinate (b2) --
					cycle;
		\node at (8,1) {=};
		\draw (9,0) coordinate (a3) --
					(11.5,0) coordinate (c3) --
					(13.5,1) coordinate (d3) --
					(10.5,2) coordinate (b3) --
					cycle;
	\end{tikzpicture}
	\caption{Die Addition zweier geometrischer Objekte (Dreiecke) ergibt ein anderes geometrisches Objekt. Dieses Bild ist nur eine symbolische Skizze dessen, womit sich die geometrische Algebra auseinandersetzt, die hier gezeigte Gesetzmäßigkeit ist nicht zwingend richtig.}
	\label{fig:addition-dreiecke}
\end{figure}
Weiterhin kann noch keine Aussage darüber getroffen werden, ob der $+$ operator in Abbildung \ref{fig:addition-dreiecke} kommutativ ist oder nicht. Solche Eigenschaften zu klären ist auch Teil der Geometrischen Algebra, da diese Eigenschaften genutzt werden können um mathematische Ausdrücke zu vereinfachen. Wie in anderen Teilgebieten der Mathematik werden bekannte wahre Aussagen verwendet von denen sich andere Aussagen oder mathematische Ausdrücke ableiten lassen. So wie in der Booleschen Algebra auf Basis der drei booleschen Operatoren komplexe Systeme aufgebaut werden können, kann in der geometrischen Algebra mithilfe der von Euklid formulierten Geometrie und algebraischen Operatoren die Physik modelliert werden.
