\documentclass[a4paper]{article}

\usepackage[ngerman]{babel}
\usepackage{amsmath}

\title{Euklidische Geometrie}
\author{Michael Hohenstein}
\date{\today}

\begin{document}

	\maketitle
	\newpage

	\section{Die Geometrie nach Euklid}

	Circa 300 Jahre vor Christus formulierte der griechische Mathematiker \textbf{Euklid von Alexandria} die mathematische Formulierung zur Beschreibung der Zwei- und Dreidimensionalen Geometrie. Seine Grundlage basiert auf 23 Definitionen, 5 Axiomen und 5 Postulaten. Die Definitionen führen ein paar Grundbegriffe, wie Punkt, Linie und Ebene ein und sind als Worteinführungen zu verstehen. Die Postulate und Axiome beschreiben die eigentlichen Eigenschaften der Euklidischen Geometrie.

	\subsection{Definitionen}
	\begin{enumerate}
		\item Ein \textbf{Punkt} hat keine Ausdehnung
		\item Eine \textbf{Linie} hat nur eine Länge aber keine Breite
		\item Die Enden einer Linie sind Punkte
		\item Eine Linie, die gleichmäßig zu den Punkten auf ihr liegt, ist eine \textbf{Gerade}
		\item Eine \textbf{Fläche} hat eine Länge und Breite
		\item Die Enden einer Fläche sind Linien
		\item Eine \textbf{Ebene} ist eine Fläche, die gleichmäßig zu den Linien auf ihr liegt
		\item Als \textbf{Ebener Winkel} wird die Neigung zweier Linien zueinander in einer Ebene bezeichnet
		\item Sind die Linien Geraden, heißt der Winkel \textbf{geradlinig}
		\item Sind alle Winkel an zwei sich schneidenden Geraden gleich, so sind die Winkel \textbf{Rechte}. Jede dieser Geraden heißt \textbf{senkrecht} zu der, auf der sie steht.
		\item Ein Winkel, der größer als ein Rechter ist, heißt \textbf{Stumpf}
		\item Ein Winkel, der kleiner als ein Rechter ist, heißt \textbf{Spitz}
		\item Eine \textbf{Grenze} beschreibt das Ende von einer Linie / Fläche
		\item Als \textbf{Figur} wir alles bezeichnet, was von Grenzen umfasst wird
		\item Ein \textbf{Kreis} ist eine ebene Figur die von einer Linie umfasst wird, sodass alle Strecken, die von einem Punkt im Inneren bis zur Linie laufen, gleich sind.
		\item  Der Punkt in der Mitte des Kreises heißt \textbf{Mittelpunkt}
		\item Zum \textbf{Durchmesser} eines Kreises gehört jede Strecke, die durch den Mittelpunkt geht und auf beiden Seiten vom Kreisumfang begrenzt ist. Eine solche Strecke halbiert den Kreis in zwei gleiche Teile.
		\item Der \textbf{Halbkreis} ist die vom Durchmesser und dem durch ihn abgeschnittenen Bogen umfasste Figur
		\item Figuren, die von Strecken umfasst sind, heißen \textbf{Geradlinig}: \textbf{Dreiseite} bestehen aus 3 Strecken, \textbf{Vierseite} bestehen aus 4 Strecken, \textbf{Vielseite} bestehen aus mehreren umfassenden Strecken
		\item Unter den Dreiseiten hat das \textbf{gleichseitige Dreieck} drei gleiche Seiten, das \textbf{gleichschenklige Dreieck} nur zwei gleiche Seiten und das \textbf{ungleichseitige Dreieck} drei verschiedene Seiten
		\item Unter den Dreiseiten gibt es auch das \textbf{rechtwinklige Dreieck}, das einen rechten Winkel hat, das \textbf{stumpfwinklige Dreieck} mit einem stumpfen Winkel und das \textbf{spitzwinklige Dreieck} das 3 spitze Winkel hat.
		\item Unter den Vierseiten ist das \textbf{Quadrat} sowohl gleichseitig als rechtwinklig. Ein \textbf{Rechteck} ist rechtwinklig, aber nicht Gegenwinklig, ohne rechtwinklig oder gleichseitig zu sein. Die andern Vierseite heißen \textbf{Trapeze}
		\item \textbf{Parallelen} sind Geraden, die in derselben Ebene liegen und sich auch bei Verlängerung nach beiden Seiten ins Unendliche nicht treffen.
	\end{enumerate}

	\subsection{Axiome}
	\begin{enumerate}
		\item Dinge, die zu dem selben gleich sind, sind auch untereinander gleich
		\item Wenn man zu gleichen Dingen Gleiches hinzufügt, ist das Ergebnis gleich
		\item Wenn man von gleichen Dingen Gleiches abzieht, sind die Reste gleich
		\item Dinge, die einander decken, sind gleich (Kongruenz)
		\item Das Ganze ist größer als ein Teil davon
	\end{enumerate}

	\subsection{Postulate}
	\begin{enumerate}
		\item Man kann von jedem Punkt aus zu jedem anderen Punkt eine Linie ziehen
		\item Begrenzte Linien können zu einer Geraden verlängert werden
		\item Wenn man einen Mittelpunkt und einen Abstand vorgibt, kann man daraus einen Kreis zeichnen
		\item Alle rechten Winkel sind gleich groß
		\item Wenn bei einer Geraden, die zwei andere Geraden schneidet, die Summe der beiden Innenwinkel (Nachbarwinkel) an der gleichen Seite kleiner ist als die Summe von zwei rechten Winkeln, so werden sich die beiden Geraden auf der Seite schneiden, an der sich diese beiden Winkel befinden.
	\end{enumerate}

Das letzte Postulat, auch als \textit{\glqq{}Parallelpostulat\grqq{}} bekannt, lässt sich auch einfacher formulieren: Wenn ein Punkt außerhalb einer Geraden vorgegeben ist, dann gibt es genau eine weitere Gerade die zu der vorgegebenen Gerade parallel ist und durch diesen Punkt verläuft. Das fünfte Postulat ist etwas länger und komplizierter formuliert als die anderen, weshalb versucht wurde, es aus den anderen 4 Postulaten herzuleiten und damit überflüssig zu machen. Jedoch ist es Gauß. Lobatschewski und Bolyai gelungen, zu zeigen, dass die euklidische Geometrie ohne dieses Postulat unvollständig ist.

\subsection{Nicht-Euklidische Geometrie}
In der Hyperbolischen Geometrie lautet das 5. Postulat:
\begin{quote}
Es gibt zu einer Geraden mehrere parallele Geraden, die durch einen vorgegebenen Punkt außerhalb der Geraden verlaufen.
\end{quote}
In einem hyperbolischen Raum hat das Parallelpostulat keine Bedeutung, da es beliebig viele Parallelen geben kann. Bei einer elliptischen Geometrie lautet das fünfte Postulat:
\begin{quote}
Es gibt zu einer Geraden keine parallele Gerade, die durch einen Punkt außerhalb der Geraden verläuft.
\end{quote}
Dies kommt daher, dass auf einer Kugeloberfläche alle parallelen übereinander liegen müssen oder sich in mindestens einem Punkt schneiden. Weiterhin sind Winkelsummen von Dreiecken und anderen Vielseiten in einem Nicht Euklidischem Raum nicht vorgegeben.

\end{document}
